Today, any non-trivial application requires the ability to communicate over the
network. Providing a secure connection for the application to achieve its goals
is a difficult task as it involves correctly implementing complex protocols.
Further, even if you could provide integrity and confidentiality of data
received over the network, it is sometimes difficult to verify the benign
nature of such data. Having stood the test of time as being the most popular
application for network communication, browsers have been able to achieve
network security with greater success. However, almost all other non-browser
applications have lagged behind. We propose that these applications delegate
network security to browsers; we built a prototype version of \emph{wget} to
demonstrate the feasibility of building an application without requiring a deep
understanding of the complexity of the TLS protocol.

We further take a closer look at package managers as they require more than
just standard network security. In particular, developers often download and
execute code from untrusted entities. Downloading and running packages over
HTTPS is not sufficient in this case. We argue for a more secure package
manager, one that can cope with nation state adversaries (who have a history of
infiltrating codebases). We describe the design of one such secure
system---SPAM---that uses the new Stellar federated Byzantine fault tolerant
system.
