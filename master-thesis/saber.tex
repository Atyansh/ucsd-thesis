\chapter{Saber: delegating transport layer security to browsers}

\section{Introduction}
\label{sec:intro-saber}

Transport Layer Security (TLS) is a cryptographic protocol designed to provide
end-to-end communication security between two parties, even in the presence of
active man-in-the-middle adversaries. It has since become the standard for
providing secure communication on the Internet and is used for several
sensitive applications such as banking, journalism, social media, shopping,
etc. Usage of TLS is most commonly seen in securing web traffic in the form of
HTTPS. When implemented correctly, HTTPS guarantees privacy, integrity, and
authenticity of all web communication between two parties. Further, there are
additional mechanisms such as \emph{strict transport security} (HSTS),
\emph{public key pinning} (HPKP), \emph{online certificate status protocol}
(OCSP) stapling, and \emph{certificate transparency} (CT), that improve the
security of HTTPS. However implementing HTTPS and all the supporting security
mechanisms correctly has proven to be a challenge for most applications.

While browsers are not free from TLS bugs, they have been patched over time
through extensive amounts of penetration testing. They have further been the
pioneers of modern HTTPS standards and are the first to implement them into
practice. Previous work has found that non-browser applications in particular
have struggled with validating TLS certificates\cite{dangerous}. We also
further find that they lag behind in other web security practices such as HSTS,
HPKP, and certificate revocation checking. Despite this, usage of non-browser
software such as package managers, wget/curl, and language specific request
libraries, is prominent. These applications and libraries support TLS and hence
provide the impression of giving security to the user, but they fail to live up
to that promise. We find that browsers such as Chrome and Firefox implement not
only secure validation of TLS certificates in general, implement mechanisms
such as HSTS, HPKP, and verification of revoked certificates, but also have
dynamic protection against identified malware and dangerous binaries. They are
also better at displaying security warnings to the user in a transparent
manner.

The issues with improper implementation of TLS for non-browser applications are
due to both the complexity of the protocol, lack of general understanding of
TLS, and not having the same security expertise and engineering resources for
development as browsers do. Solving these issues would be a monumental task and
would require widespread education of the importance of properly implementing
communication security, along with the expense of recruiting security experts
for every application that requires using TLS, which may not be feasible for
several applications that are maintained by individual developers but are
nonetheless popular.

We propose a new way to build non-browser applications that allows developers
to use the TLS protocol for web security but does not require the expertise
needed to implement TLS verification correctly. Our method involves delegating
the handling of connection security to browsers, so the non-browser application
deals only with application layer logic. This allows non-browser applications
to get connection security for free from browsers. This also lets them take
advantage of any mechanisms that browsers implement to harden TLS security
(such as HSTS, HPKP, etc.) for free as well. We have built a prototype version
of wget using this method as a proof of concept. Our application is able to
validate proper TLS connections without writing any code that requires a deep
knowledge of the TLS protocol.

This chapter is organized as follows: In \Cref{sec:background-saber}, we
discuss the background material related to TLS and HTTPS, the various existing problems
with the ecosystem and how browsers have taken steps to solve them. In
\Cref{sec:problems-saber} we look at some non-browser applications, in particular
\emph{wget} to see how they compare to modern browsers in terms of connection security.
In \Cref{sec:saber-saber}, we present the design of a library that delegates
connection security to the browser. In \Cref{sec:swget-saber}, we present \emph{swget} as
a prototype implementation of wget that provides better connection security by delegating
TLS to browsers. In \Cref{sec:conclusion-saber}, we conclude. In \Cref{sec:related-saber},
we discuss related work.
% TODO: write sections descriptions better


\section{Background}
\label{sec:background-saber}



\section{Problems}
\label{sec:problems-saber}



\section{Solution}
\label{sec:saber-saber}



\section{swget}
\label{sec:swget-saber}

\section{Conclusion}
\label{sec:conclusion-saber}

\section{Related Work}
\label{sec:related-saber}
